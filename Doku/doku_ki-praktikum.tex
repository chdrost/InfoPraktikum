% !TEX encoding = UTF-8 Unicode
%Präambel

%Report für große Doukumente. Dieser ist in Kapitel (\chapter{}) aufgeteilt
%\documentclass[12pt, a4paper, ngerman]{report} 

%Article für normale Doumente
\documentclass[12pt, a4paper, ngerman]{article}

%Deutsche Beschreibungen von generiertem Text (table of contents => Inhaltsverzeichnis)
\usepackage[ngerman]{babel}

%Umlaute
\usepackage[utf8]{inputenc}

%Schriftart Helvetica 
\usepackage[scaled]{helvet}

%Seitenränder
\usepackage{geometry}
%top = Abstand nach oben
%left = Abstand nach links
%right = Abstand nach rechts
%bottom= Abstand nach unten
%heapsep= Abstand zwische Kopfzeile und Text
%footskip= Abstand zwischen Text und Fußzeile
\geometry{a4paper, top=25mm, left=30mm, right=25mm, bottom=30mm, headsep=10mm, footskip=12mm}

%Farben nutzen
\usepackage{xcolor}

%Grafiken einbinden
\usepackage{graphicx}

%Zusätzliche Positionsbefehle
\usepackage{float} 

%Die Einrücktiefe bei einem neuen Absatz
\setlength{\parindent}{0pt}

%Fülltext
\usepackage{blindtext}

%Fuer Zitate	
\PassOptionsToPackage{backend=bibtex}{biblatex}
\usepackage[natbib=true,style=numeric]{biblatex}
\usepackage[babel,german=guillemets]{csquotes}
\bibliography{quellen.bib} 

% Aufnahme von \paragraph in das Inhaltsverzeichnis 
\setcounter{tocdepth}{3}  

%Nummerierung vertiefen, \paragraph kommt mit ins Inhaltsverzeichnis
\setcounter{secnumdepth}{4} 

%Feste Tabellen
\usepackage{tabulary}

%caption für nummerierte Tabellenüberschriften
%booktabs für die Steuerung von Linien
\usepackage{caption, booktabs}

%Eigene Kommandos
% Osi Modell
\newcommand{\osi}{ISO/OSI Referenzmodell\xspace}
\newcommand{\fcs}{FCS, Frame Checking Sequence,\xspace}
\newcommand{\punkt}[1]{\begin{itemize} \item #1 \end{itemize}}


\usepackage{bytefield}

%Ende Präambel
	
\begin{document}

\begin{titlepage}
		\begin{center}
			\includegraphics[width=.8\linewidth]{Grafiken/logo_htw.jpg}\\[1cm]    
			\textsc{\LARGE Hochschule für Technik und Wirtschaft \newline Fakultät für Ingenieurwissenschaften}\\[1.5cm]
			\newcommand{\HRule}{\rule{\linewidth}{0.5mm}} \HRule \\[0.4cm] { \huge \bfseries Dokumentation des Ki Praktikums}\\[0.4cm]
			\HRule \\[1.5cm]

			\begin{minipage}{0.4\textwidth}
				\begin{flushleft} \large
					\emph{Projektleiter:} \\
					\emph{Team 1:}Christoph Drost\\
					\emph{Team 2:} Deniz Kadiogullari
					\end{flushleft}
			\end{minipage}
			\hfill
			\begin{minipage}{0.4\textwidth}
				\begin{flushright} \large
					\emph{Betreuer:} \\
					Prof. Dr. Weber \\
				\end{flushright}
			\end{minipage}
			\vfill
			{\large \today}
		\end{center}
	\end{titlepage}


%Inhaltsverzeichnis auf eigener Seite
\tableofcontents
\newpage 

\section{Aufbau des Datensatzes}

Der Datensatz wurde in Form einer CSV Datei zur Verfügung gestellt. Er besteht aus 45 Spalten, die jeweils die Attribute eines Satzes zur beschreiben. \\ \\

\begin{bytefield}[bitwidth=4.1em]{8}
	\bitheader{0-7} \\
	\begin{rightwordgroup}{01}
		\bitbox{1}{00} & \bitbox{1}{01} & \bitbox{1}{02}&\bitbox{1}{03} & \bitbox{1}{04} & \bitbox{1}{05}\bitbox{1}{06} & \bitbox{1}{07}
	\end{rightwordgroup} \\
	\begin{rightwordgroup}{02}
		\bitbox{1}{08} & \bitbox{1}{09} & \bitbox{1}{10}&\bitbox{1}{11} & \bitbox{1}{12} & \bitbox{1}{13}\bitbox{1}{14} & \bitbox{1}{15}
	\end{rightwordgroup} \\
	\begin{rightwordgroup}{03}
		\bitbox{1}{16} & \bitbox{1}{16} & \bitbox{1}{17}&\bitbox{1}{18} & \bitbox{1}{19} & \bitbox{1}{20}\bitbox{1}{21} & \bitbox{1}{21}
	\end{rightwordgroup} \\
	\begin{rightwordgroup}{04}
		 \bitbox{1}{22} & \bitbox{1}{23}&\bitbox{1}{24} & \bitbox{1}{25} & \bitbox{1}{26}\bitbox{1}{27} & \bitbox{1}{28} & \bitbox{1}{29} 
	\end{rightwordgroup} \\
	\begin{rightwordgroup}{05}
		\bitbox{1}{30} & \bitbox{1}{31} & \bitbox{1}{32}&\bitbox{1}{33} & \bitbox{1}{34} & \bitbox{1}{35}\bitbox{1}{36} & \bitbox{1}{37}
	\end{rightwordgroup} \\
	\begin{rightwordgroup}{06}
		\bitbox{1}{38} & \bitbox{1}{39} & \bitbox{1}{40}&\bitbox{1}{41} & \bitbox{1}{42} & \bitbox{1}{43}\bitbox{1}{44} & \bitbox{1}{45}
	\end{rightwordgroup} \\
\end{bytefield}
	
\textbf{Erklärung}
\begin{enumerate}
	\item Loc\_Id, eindeutiger numerischer Wert des Datensatzes.
	\item Gebietstyp. Dadurch wird das Gebiet des Datensatzes beschrieben. Bsp. A3 für ein eigenes Land, L1 für Autobahn.
	\item  Untertyp des Gebiets, dient dazu, das Gebiet genauer zu klassifizieren.
	\item Kurzbezeichnung für Straße, Bsp: A620.
	\item Straßenname, bzw. Gebietsname. Bsp: Messegelände, Ruhrallee
	\item Erster Straßenname. Bsp: Essen-Heisingen ? von ?
	\item Zweiter Straßennahme ? Bis ?????
	\item Area-Verweis (über Loc\_Id) auf das Gebiet, in dem die Lokation liegt
	\item Linear Reference ???????????
	\item Verweis auf Vorgängerlokation (Loc\_Id) bezogen auf die Erfassungsrichtung
	\item Verweis auf Nachfolgerlokation (Loc\_Id) bezogen auf die Erfassungsrichtung
	\item Flag, ob Verkehr städtischen Charakters vorliegt (1=ja, 0=nein)
	\item Verweis auf Lokation (Loc\_Id) einer anderen Straße an gleicher Stelle (zirkuläre Verkettung)
	\item Verweis auf die nächste Lokation nach der Unterbrechung im Staßenverlauf (wahrscheinlich Loc\_Id)
	\item Flag, ob Lokation in Erfassungsrichtung zugänglich ist (1=ja, 0=nein)
	\item Flag, ob Lokation in Erfassungsrichtung verlassen werden kann (1=ja, 0=nein)
	\item Flag, ob Lokation entgegen der Erfassungsrichtung zugänglich ist (1=ja, 0=nein)
	\item Flag, ob Lokation entgegen der Erfassungsrichtung verlassen werden kann (1=ja, 0=nein)
	\item Flag, ob Lokation in Erfassungsrichtung vorhanden ist (1=ja, 0=nein)
	\item Flag, ob Lokation entgegen der Erfassungsrichtung vorhanden ist (1=ja, 0=nein)
	\item Anschlussstellennummer/Knotenpunktnummer (bei BAB)
	\item Nummer der in Erfassungsrichtung ausgewiesenen Umleitungsempfehlung
	\item Nummer der entgegen der Erfassungsrichtung ausgewiesenen Umleitungsempfehlung
	\item Flag, ob Datensatz bei Aktualisierungslauf gegenüber der Vorgängerversion verändert wurde (nur vor Release erkennbar 0=nein, 1=ja, 2=neu, 3=löschen)
	\item Angabe, ob Lokation zum TERN-Netz gehört (1=ja, 0=nein)
	\item Netzknotennummer der Lokation oder Netzknotennummer des vor der Lokation liegenden Netzknoten
	\item Netzknotennummer des nach der Lokation liegenden Netzknoten
	\item Entfernung der Lokation vom Netzknoten 1 (Angabe in m) in Richtung Netzknoten 2
	\item Angabe der x-Koordinate nach WGS84 ohne vorangestellte Nullen (s. EN ISO 14819-3)
	\item Angabe der y-Koordinate nach WGS84 ohne vorangestellte Nullen (s. EN ISO 14819-3)
	\item Hinweis auf zuständige Polizeidienststelle (Nur Name)
	\item Hinweis auf für Bearbeitung zuständiges Bundesland (Kurzzeichen nach StBA) 
	
	
\end{enumerate}	

Die Felder 31-44 sind nicht dokumentiert.

\subsection{Erster Versuch der Interpretation}
Lokation: Saarbrücken-Malstatter Brücke

\textbf{Vorläufig wichtige Attribute}

Geo Koordinate: X: 00697225; Y: 4923550  (Google: ca. 49.235514, ca. 6.971922)

Area Verweis: 2291 = Saarbrücken

Linear Reference:  7172 =  First Name = Saarlouis ; Second Name = Saarbrücken evtl. verweist die Linear Reference auf 

\newpage
\listoffigures
\end{document}